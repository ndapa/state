%
% File acl2012.tex
%
% Contact: Maggie Li (cswjli@comp.polyu.edu.hk), Michael White (mwhite@ling.osu.edu)
%%
%% Based on the style files for ACL2008 by Joakim Nivre and Noah Smith
%% and that of ACL2010 by Jing-Shin Chang and Philipp Koehn


\documentclass[11pt,a4paper]{article}
\renewcommand{\baselinestretch}{1.035}
\usepackage{acl2014}
\usepackage{times}
\usepackage{latexsym}
\usepackage{amsmath}
\usepackage{color}
\usepackage{epsfig,url,algorithm,algorithmic,multirow}
%\usepackage{microtype}
\usepackage{amssymb}


%for writing 
%\usepackage{amsfonts}
%\usepackage{amssymb}
%\usepackage{calligra}
%\usepackage{calrsfs}
%\usepackage[left=2cm,right=2cm,top=2cm,bottom=2cm]{geometry}
%\usepackage[mathscr]{euscript}


%%EXTRA HOME-GROWN MACROS

\newcommand{\squishlist}{
 \begin{list}{$\bullet$}
  { \setlength{\itemsep}{0pt}
     \setlength{\parsep}{3pt}
     \setlength{\topsep}{3pt}
     \setlength{\partopsep}{0pt}
     \setlength{\leftmargin}{1.5em}
     \setlength{\labelwidth}{1em}
     \setlength{\labelsep}{0.5em} } }

\newcommand{\squishlisttwo}{
 \begin{list}{$\bullet$}
  { \setlength{\itemsep}{0pt}
     \setlength{\parsep}{0pt}
    \setlength{\topsep}{0pt}
    \setlength{\partopsep}{0pt}
    \setlength{\leftmargin}{2em}
    \setlength{\labelwidth}{1.5em}
    \setlength{\labelsep}{0.5em} } }

\newcommand{\squishend}{\end{list}}

\newtheorem{theorem}{Theorem}
\newtheorem{lemma}{Lemma}
\newtheorem{corollary}{Corollary}
\newtheorem{definition}{Definition}
\newtheorem{example}{Example}
\newtheorem{hypothesis}{Hypothesis}
%\newtheorem{hypothesis}{Hypothesis}[section]
\newcommand{\newWord}[1]{\emph{#1}}
\newcommand{\ignore}[1]{}
%\newcommand{\comment}[1]{{\color{red}{#1}}}
\newcommand{\comment}[1]{}
%\newcommand{\remark}[1]{{\color{red}{#1}}}
\newcommand{\indented}[1]{
	\begin{tabbing}
		blah\=\+\kill
		\parbox{8cm}{#1}
	\end{tabbing}
}
\renewcommand{\paragraph}[1]{\noindent\textbf{#1.}}
\newcommand{\ruleline}[1]{\indented{#1}}



\newcommand{\fact}{{\footnotesize \texttt}}
\newcommand{\factid}[2]{\emph{#1:} \texttt{#2}}
\newcommand{\factspo}[3]{{\footnotesize $<$\texttt{#1}, \texttt{#2}, \texttt{#3}$>$}}
\newcommand{\factidspo}[4]{\emph{#1:} \texttt{#2} \texttt{#3} \texttt{#4}}
\newcommand{\spotl}[1]{\texttt{<#1>}}
\newcommand{\spotlfive}[5]{$\langle$\texttt{#1}, \texttt{#2}, \texttt{#3}, [\texttt{#4}], $\mathtt{#5}\rangle$}
\newcommand{\entity}[1]{{\footnotesize \texttt{#1}}}
%\newcommand{\entity}{{\footnotesize \texttt}}
\newcommand{\rel}[1]{{\footnotesize \texttt{#1}}}

% Fabian: Use this to define our pattern language
\newcommand{\instance}[1]{\textit{$\langle$#1$\rangle$}}
\newcommand{\pos}[1]{\textit{#1}}
\newcommand{\pattern}[1]{\textit{#1}}
\newcommand{\str}[1]{``#1''}
\newcommand{\svo}{\pattern{{\instance{S}}V{\instance{O}}}}

% Fabian: Choose second option to remove superfluous stuff for space reasons
\newcommand{\unimportant}[1]{#1}
%\newcommand{\unimportant}[1]{}

% Fabian: for full-text words
% I find texts that use different styles for different types of words very unreadable.
% usually one style is sufficient.
\newcommand{\word}[1]{``#1''}

% hyphen
\hyphenation{Wi-ki-pe-dia}

%%%%%%%%%%%% system
\newcommand{\sellname}{FactChecker}

\newcommand{\highlight}[1]{{\color{red} #1}}

%\usepackage{acl-hlt2011}
\usepackage{times}
\usepackage{latexsym}
\usepackage{amsmath}
\usepackage{multirow}
\usepackage{url}
\DeclareMathOperator*{\argmax}{arg\,max}

\usepackage{booktabs}
\usepackage{graphicx}
\usepackage{varwidth}

%% END EXTRA HOME-GROWN MACROS

% take from http://tex.stackexchange.com/questions/40283/wrapping-table-column-headings-in-turn-environment
\newcommand{\turn}[3][10em]{% \turn[<width>]{<angle>}{<stuff>}
  \rlap{\rotatebox{#2}{\begin{varwidth}[t]{#1}#3\end{varwidth}}}%
}

%\title{FactChecker: Leveraging Language and Co-mentions for\\
\title{Contextual Temporal Profiles\\ for State Change Detection in Knowledge Bases}
\author{Derry, Ndapa, Tom\\
Carnegie Mellon University\\ 5000 Forbes Avenue\\Pittsburgh, PA\\
  { \small
    \{ndapa\}@cs.cmu.edu}
}

\author{
%1st Author and 2nd Author\\
%Organization\\
%City, Country\\
%  { \small
%    email addresses}
}

\begin{document}

\maketitle

%\vspace{-1cm} %ineffective here anyway

\begin{abstract}
 Methods for information extraction (IE) and knowledge base (KB)
construction have been widely studied in recent years.  However,   maintenance of  knowledge once it has been acquired is still largely under-explored. In real-life, changes happen to entities in the knowledge base. Propagating  these changes to the KB is crucial for reducing inaccuracies in the  state of the KB. 
In this paper, we present a method for detecting 
state changes of KB entities.
We introduce a semi-supervised method that leverages Contextual Temporal Profiles (CTPs) of  entities to detect change inducing contexts. Our  experiments show how our method can be used to successful  detect the time when an entity undergoes a certain state change.
% for detecting state changes and hence recommending updates to specific attributes of entities in the KB.

\end{abstract}
%%%%%%%%%%%%%%%%%%%%%%%%%%%%%%%%%%%%%%%%%%%%%%%%%%%%%%%%%%%%%%
\section{Introduction}
%%%%%%%%%%%%%%%%%%%%%%%%%%%%%%%%%%%%%%%%%%%%%%%%%%%%%%%%%%%%%%
%Motivation
%\paragraph{Motivation}
%
Knowledge Base (KB) construction is at a point where  problems related to maintenance can no longer be avoided.   One such problem is the detection of state changes that KB entities undergo overtime. When the relation value of a given entity is no longer true, the KB needs to be updated.  This can occur for example if a person  gets divorced or a company undergoes management changes.  A consequence of the state change person  fired or resigns from their job,  is that the person is no longer an employee of their current employer. Currently, most IE methods detect patterns for learning attributes and mix them together with those involving state changes in the attribute.  It is not unusual for an IE system to learn that the phrases: ``is married to" and ``is divorced from"  are both high precision patterns to indicate the \textit{hasSpouse} relation. In reality,  one  of those phrases marks the beginning and the other marks the end of an relation value for the \textit{hasSpouse} relation.  \highlight{To DO: give specific defn of state change in terms of predicates.}\ignore{Specifically define what a state change is, change Given a relation R, an entity e, state change is when value acquires a new entity}

%\paragraph{State-of-the-Art and Open Questions} 
State change is related to KB base update in general. However, prevalent approaches to IE extract knowledge from static Web snapshots \cite{Fader11,Nakashole11}, with no mention of how to perform updates. Other approaches periodically extract knowledge from a corpus such as Wikipedia, every time re-applying the extractor to all the documents even those that did not change \cite{Suchanek07}.  The NELL system \cite{Carlson10} follows a never-ending extraction model with the extraction process going on 24 hours a day. However NELL's focus is on language learning  to self improve on its reading ability over time.  In contrast, here we focus on  detecting  updates to specific attributes of entities.  Detecting these changes has unique   challenges not seen in IE methods: 
  i) The state of an entity in the knowledge base can change in two ways: a) the entity  acquires a new attribute value (e.g, person is newly  married or has a new job)
 b) the entity loses an attribute value (e.g, person is divorced or was fired). Knowledge acquisition methods focus on, by design,  acquiring  new attribute values but not on detecting loss of relation values.  In contrast, the problem of state change detection includes detection of \textit{both}
 acquisition and loss of relation values.
 ii)  In state change detection, we would like to detect the state change at the right time.
 % so that the facts in the KB are correct soon after the state change takes place in the real world
  This means that state change detection requires accurate and timely  temporal  scoping of  relation values.
  % State-of-the-art KBs are lacking in terms of temporal information. 
  
%\item \textbf{Targeted extraction model:}  IE methods often  extract everything they are able to find in a given corpus. However, to capture state changes, we need to instead employ a targeted extraction model. This model needs to find documents that  are promising for state change information for  given entity. A method that identifies the documents of interest can leverage  news aggregation and micro news platforms such as Twitter.
%\end{enumerate}

In this paper we propose a method for state change detection, based on Contextual Temporal Profiles (CTPs)
of KB entities.
%Our method can identify both attribute value acquisition and loss.
%Furthermore, our method is able to produce accurate temporal scopes of attribute values.
%Our solution is a semi-supervised method that leverages a time-stamped corpus spanning several decades. 

%\paragraph{Contribution  and Paper Organization} 
%We developed a method for detected state changes. Our approach finds  state change sequences across temporal profiles of similar entities. \highlight{TO DO: describe method fully, followed by overview of the rest of the paper.}

%%%%%%%%%%%%%%%%%%%%%%%%%%%%%%%%%%%%%%%%%%%%%%%%%%%%%%%%%%%%%%
\section{Method}
%%%%%%%%%%%%%%%%%%%%%%%%%%%%%%%%%%%%%%%%%%%%%%%%%%%%%%%%%%%%%%
Our main idea is that given an entity, and its Contextual Temporal Profile (CTP), we can learn 
when such an entity undergoes  specific state changes. The CTP at a given time point $t$ contains the context  within which the entity is mentioned at that time . Our method is based on two related insights: i)  the context of the entity at time $t$  reflects the state change that the entity undergoes.  ii) the   \textit{difference in context} before, at time $t-1$, and after, at time $t+1$,  reflects the context responsible for the state change at time $t$.  However naively applying these two insights does not result in a good solution. This is because the entity can undergo a multiplicity of changes at the same time  $t$. Thus both the contexts and the differences in contexts can contain information  pertaining to several state changes. We therefore need a way of determining which part of the context is relevant to a given state change $sc_i$. To this end, we generate what we refer to as an  $aggregate$ CTP, $CTP(\hat{e}, sc_i)$   for a hypothetical average entity $\hat{e}$ undergoing state change $sc_i$.  We generate $\hat{e}$ and its $CTP(\hat{e}, sc_i)$ for from the CTPs  of  a seed set of entities that undergo state change $sc_i$. Once aggregate CTPs are computed, we can then use them to detect state changes of previously unseen entities by their similarity to the aggregate CTPs.
%  Thus the context of  $\overline{E}$'s CTP at time $t$ is derived from the contexts of all the seed entities for state change $sc_i$.


\subsection{Learning  State Change Vectors}
To build CTPs for entities, we use the Google Books Ngram corpus.  It contains n-grams for up to $n=5$, along with occurrence statistics from over about 5 million digitized books \cite{Michel11}. Of importance is the fact that this corpus is time-stamped, with granularity of  year. 

\begin{definition}[\textbf{Contextual Temporal Profile}]
The Contextual Temporal Profile (CTP) of an entity $e$, at time $t$ consists of
the context, $C_e(t)$, within which $e$ is mentioned. Specifically $C_e(t)$  is generated from the  5-grams that mention $e$ at time $t$.\ignore{Ndapa:uni or bigrams??}  Computing all contexts $C_e(t)$, $\forall t_i$ generates the CTP of $e$.
\end{definition}

The CTPs can contain contextual unit ( uni-grams or bi-grams) that are not related to state change (s) occurring at time $t$. We therefore would like to ensure that the context is free of such words. The first thing we do is to disregard stop words. The second thing is to compute $tf-idf$ statistics for each contextual unit  and only retain the top $k$ ranking units in the context $C_e(t)$. We compute tf-idf by  treating each time unit $t$ as a document containing words that occur in the context of $e$.

Additionally, notice that the  CTPs may contain contexts attributed  to several state changes. We therefore tease apart the CTPs to isolate contexts specific to a given state change. For this, our method takes as input  a small set of seed entities, $\mathcal{S}(sc_i)$, for each type of state change . Thus for  US presidency state change, where an entity enters or leaves the presidency, we would have seeds as follows: for entering office,  \textit{(Richard Nixon, 1969),(Bill Clinton, 1993)} and for leaving office \textit{(Richard Nixon, 1974),(Bill Clinton, 2001)}. From the CTPs of the seeds for state change $sc_i$, we generate an aggregate  CTP, $CTP(\overline{e}, sc_i)$.

\begin{definition}[\textbf{ Aggregate CTP for $\hat{e}$}]
The CTP of a mean entity $\hat{e}$, at time $t$, for state change $sc_i$ is made up of the contexts of the CTPs of all entities in the seed set  $\mathcal{S}(sc_i)$ that undergo state change $sc_i$ at time $t$.
Thus,  the aggregate context $C_{\overline{e}}(t)$ = $\bigcap_{ e : C_e(t) \neq \emptyset } C_{e}(t) $. 

% at  respective $t$ $C_e(t)$, within which $e$ is mentioned. Specifically $C_e(t)$  is made up of all  5-grams that mention $e$ at time $t$. Computing all contexts $C_e(t)$ $\forall t_i$ generates the CTP of $e$.
\end{definition}

Notice that so  far  the aggregate CTPs $CTP(\overline{e}, sc_i)$, a non-empty aggregate   contexts  only appears at time points $ti, tx, tw, ... tz$ where at least one  entity in the seed set, $\mathcal{S}(sc_i)$, undergoes a state change. For reasons that will become clear later,  we also generate aggregate contexts for times $t-1$ and $t+1$, in the same way we generate $C_{\overline{e}}(t)$. A distinction  is made between contexts at times $t-1$, $t$, $t+1$ and the way we treat them. 

One might think that the aggregate CTP $CTP(\overline{e}, sc_i)$ is likely to be sparse in terms of populated time points, depending on the number of seeds. However, the way we use the profiles is such that we generate vectors from these CTPs.  In particular we generate three vectors $v_c, v_{bc}, v_{ac}$.  

\begin{definition}[\textbf{  Aggregate Change Vectors}]
From the aggregate CTP, $CTP(\overline{e}, sc_i)$, we generate three change vectors, $v_c, v_{bc}, v_{ac}$:  $v_c$ reflects the context during the state change, $v_c = \cup_{w\in C_{\overline{e}}(t)}$; $v_{bc}$ is a difference vector containing the difference in context before and the context during change:  $v_{bc} = C_{\overline{e}}(t) - C_{\overline{e}}(t-1)  $.   $v_{ac}$ is a difference vector containing the difference in context after and the context during change:  $v_{ac} = C_{\overline{e}}(t) - C_{\overline{e}}(t-1)  $. 
\end{definition}

The aggregate change vectors are then used detecting state changes.

\subsection{Detecting State Changes}
To detect state changes for a previously unseen entity, we generate its CTP, its change vectors, compare every these info for each candidate time $t$ for state change $sc_i$ to happen. Rank the candidate time points based on their similarity to the aggregate change vectors. \highlight{ TO DO: expand this sections}
 
%For unseen entities. We compare to average of the profiles
%
%We introduce a semi-supervised approach  to state change detection.  
%As input we have provide for each attribute a few seed  tuples. A seed is  a 5-tuple in the following format: a entity, an attribute, an attribute value, valid begin time, valid end time. An example seed tuple is:  (Bill Clinton, holdsPoliticalOffice, President of US, 1993, 2001).   For each attribute-value pair, we collect a few seed tuples, e.g, 10 US presidents and their validity periods. For each entity we build a temporal profile from the time-stamped corpus. For each attribute we learned a collection of words related to that particular attribute through distant supervision. Thus for holdsPoliticalOffice we have related works like "elect, sign, bill, campaign, legislature, congress, etc". 
%For each entity we thus build a temporal profile per attribute, spanning a period of time.
%We then build a distribution of words for entities in the seed sample  during the start time of their tuple and during the end time of their tuple.
%
%Detecting attribute addition: we get a new entity, for every year for each relation we build a profile for the entity, we then determine if each year is the start year of the predict or not at all.
%This method works well for edge additions but not edge deletions. 
%
%For relation value deletions:  We look into  edge deletion.


%The approach is as follows:\\
%For every seed tuple we have a vector Vc - change- which contains the difference between the word (can also be bi-gram) distribution at time  t mines word distribution at time t-1. also another vector of change is represented by word at t+1  minus words at t.
%We have also Verb vector for the year, Vy which contains all verbs at time t, mentioned with more than k entities in the seed tuple.
%From the training data we obtain Vy mean and Vc mean.
%For a new value, we 

%\ignore{Sequence alignment problem: how to map non-tivially mapped sequences to show how casaded changes affect the entire graph when something changes. When one is fired, company hires someone else. How do we know it is not by chance that two sequences are aligned and maybe are aligned but not related.}

%\subsection{State Change Example}
%Here add an example for the presidency relations.
%Show the change in words. As well as temporal profile.
%\begin{itemize}
%\item Imagine a knowledge graph (KG), at any given time an entity has state t.
%the state changes overtime. Through edge additions and add deletions.
%
%\item  Changes can also appear through cascades, where state change of one entity, affects 
%also many. CEO may resign from a company, his state changes, so does the company, but also 
%another entity might have to step in to be the president. We do not model cascades.
%
%\item  We focus on extracting edge additions and deletions
%\end{itemize}

%\subsection{Approach Overview}
%\begin{itemize}
%\item a semi-supervised method that takes as input a few instances per relation
%and the time periods at which the training entities went through state changes with respect to the given relations.
%
%\item  makes use of the change in language used before and after the state change. For unseen entities.
%
%\item Using change in words usage over temporal profile effective for  add additions 
%\end{itemize}








%%%%%%%%%%%%%%%%%%%%%%%%%%%%%%%%%%%%%%%%%%%%%%%%%%%%%%%%%%%%%%
\section{Experiments}
%%%%%%%%%%%%%%%%%%%%%%%%%%%%%%%%%%%%%%%%%%%%%%%%%%%%%%%%%%%%%%

Experiments comparing to Cots, gradient on several relations whose
temporal scopes are relevant when it comes to state changes,.
In fact state change is nothing more than changes that occur with time, hence
temporally scoping facts in a KB is isomorphic to state change detection, except we would like
to do state change detection as it happens as opposed to retrospectively scoping facts 
once they have been harvested.


%%%%%%%%%%%%%%%%%%%%%%%%%%%%%%%%%%%%%%%%%%%%%%%%%%%%%%%%%%%%%%
\section{Related work}
%%%%%%%%%%%%%%%%%%%%%%%%%%%%%%%%%%%%%%%%%%%%%%%%%%%%%%%%%%%%%%
%Despite the importance of time in any information space,
%gathering and distilling temporal knowledge fromWeb sources
%remains a major research challenge [28]. To the best of our
%knowledge, Timely YAGO [27] and PRAVDA [26], two re-
%cently proposed techniques, are the only systems which try
%to harvest temporal facts. Timely YAGO tries to automat-
%ically scope facts using regular expressions in Wikipedia in-
%foboxes, and hence is not applicable to widely available free
%text. PRAVDA, a promising recent approach, uses a combi-
%nation of textual patterns and graph-based re-ranking tech-
%niques to harvest facts and their temporal points at the same
%time. However, it is not immediately clear how this approach
%could be used to temporally scope facts in an existing knowl-
%edge base. Other works on temporal information extraction
%have tackled Other works on temporal information extraction
%have tackled partial aspects of the problem such as tempo-
%ral relations identication between events [6, 14, 5, 16, 8, 29,
%13]. However, these other works are not sucient to tempo-
%rally scope facts as they are all focused on micro-reading of
%time at a single document or sentence level, i.e., temporal ex-
%pressions and relations are normalized and identied only on features derived from the document. However, web is unique in that it typically oers ample redundancy, and
%hence it only seems natural to aggregate many observed cues
%for a temporal fact in a statistical manner [28].


%%%%%%%%%%%%%%%%%%%%%%%%%%%%%%%%%%%%%%%%%%%%%%%%%%%%%%%%%%%%%%
\section{Conclusion}
%%%%%%%%%%%%%%%%%%%%%%%%%%%%%%%%%%%%%%%%%%%%%%%%%%%%%%%%%%%%%%
We conclude as follows:


%\newpage
%\begin{thebibliography}{ABCD99}
\begin{thebibliography}{99}

\bibitem[Angel 2012]{Angel12} A. Angel, N. Koudas, N. Sarkas, D. Srivastava: Dense Subgraph Maintenance under Streaming Edge Weight Updates for Real-time Story Identification. In \textit{Proceedings of the VLDB Endowment}, PVLDB 5(10):574--585, 2012.

\bibitem [Auer 2007]{Auer07} S. Auer, C. Bizer, G. Kobilarov, J. Lehmann, R. Cyganiak, Z.G. Ives: DBpedia: A Nucleus for a Web of Open Data.  In \textit{Proceedings of the 6th International Semantic Web Conference (ISWC)}, pages 722--735, Busan, Korea, 2007.

%\bibitem[Adler 2007]{Adler07} B. T.  Adler, L. de Alfaro: A content-driven reputation system for the wikipedia. In \textit{Proceedings  of the 16th International Conference on World Wide Web (WWW)}, pages  261-270, 2007.

\bibitem[Banko 2007]{Banko07} M. Banko, M. J. Cafarella, S. Soderland, M. Broadhead, O. Etzioni: Open Information Extraction from the Web. In \textit{Proceedings of the 20th International Joint Conference on Artificial Intelligence (IJCAI)}, pages 2670--2676, Hyderabad, India, 2007.

\bibitem[Bollacker 2008]{Bollacker08} K. D. Bollacker, C. Evans, P. Paritosh, T. Sturge, J. Taylor: Freebase: a Collaboratively Created Graph Database for Structuring Human Knowledge.  In \textit{Proceedings of the ACM SIGMOD International Conference on Management of Data (SIGMOD)}, pages, 1247-1250, Vancouver, BC, Canada, 2008.

%\bibitem[Brown 2001]{Brown01} Lawrence D. Brown, T.Tony Cai, Anirban Dasgupta: Interval Estimation for a Binomial Proportion. Statistical Science 16: pages 101--133, 2001.
%
%\bibitem[Cabrio 2012]{Cabrio2012} E. Cabrio, S. Villata. Combining Textual Entailment and Argumentation Theory for Supporting Online Debates Interaction. In \textit{Proceedings of the 50th Annual Meeting of the Association for Computational Linguistics (ACL)}, pp. 208-212, 2012.
%
%%\bibitem[Castillo 2011]{Castillo2011}C. Castillo, M. Mendoza, B. Poblete: Information credibility on twitter. In \textit{Proceedings  of the 20th International Conference on World Wide Web (WWW)}, pages 675-684, ACM, 2011.

\bibitem[Carlson 2010]{Carlson10} A. Carlson, J. Betteridge, R.C. Wang, E.R. Hruschka, T.M. Mitchell: Coupled Semi-supervised Learning for Information Extraction. In \textit{Proceedings of the Third International Conference on Web Search and Web Data Mining (WSDM)}, pages 101--110, New York, NY, USA, 2010.
		 
		 
\bibitem[Carlson 2010]{Carlson10b} A. Carlson, J. Betteridge, B. Kisiel, B. Settles, E. R. Hruschka Jr., T. M. Mitchell: Toward an Architecture for Never-Ending Language Learning.  In \textit{Proceedings of the Twenty-Fourth AAAI Conference on Artificial Intelligence (AAAI)} 2010.

\bibitem[Del Corro 2013]{DelCorro2013} L. Del Corro, R. Gemulla: ClausIE: clause-based open information extraction. In \textit{Proceedings  of the 22nd International Conference on World Wide Web (WWW)}, pages 355-366. 2013.

%\bibitem[Dong 2009]{Dong09}X. Dong, L. Berti-Equille, and D. Srivastava. Truth discovery and copying detection in a dynamic world.  In \textit{Proceedings of the VLDB Endowment} PVLDB, 2(1), pp. 562-573, 2009.
%
%%\bibitem{Cooper2012} Brian F. Cooper (Editor): Special Issue on Big Data War Stories. IEEE Data Engineering Bulleting 35(2): 2012.

\bibitem[Das Sarma 2011]{DasSarma11} A. Das Sarma, A. Jain, C. Yu: Dynamic Relationship and Event Discovery.  In \textit{Proceedings of the Forth International Conference on Web Search and Web Data Mining (WSDM)}, pages 207--216, Hong Kong, China, 2011.

\bibitem[Fader 2011]{Fader11} A. Fader, S. Soderland, O. Etzioni: Identifying Relations for Open Information Extraction. In \textit{Proceedings of the 2011 Conference on Empirical Methods in Natural Language Processing (EMNLP)}, pages 1535--1545, Edinburgh, UK, 2011.
%               		 
%	
%\bibitem[Galland 2010]{Galland2010} A. Galland, S. Abiteboul, A. Marian, P Senellart: Corroborating information from disagreeing views. In \textit{Proceedings of the 3rd International Conference on Web Search and Web Data Mining (WSDM)}, pages 131-140, 2010.
	
\bibitem[Havasi 2007]{Havasi07} C. Havasi, R. Speer,  J. Alonso. ConceptNet 3: a Flexible, Multilingual Semantic Network for Common Sense Knowledge.  In \textit{Proceedings of the Recent Advances in Natural Language Processing (RANLP)}, Borovets, Bulgaria, 2007.
	
%%\bibitem[Hellmann 2009]{Hellmann09}
%%Sebastian Hellmann, Claus Stadler, Jens Lehmann, Sören Auer: DBpedia Live Extraction. OTM Conferences (2) 2009: 1209-1223.
%	 
%%\bibitem [Hoffart 2011]{Hoffart11}	J. Hoffart, M. A. Yosef, I.Bordino and H. Fuerstenau, M. Pinkal, M. Spaniol, B.Taneva, S.Thater, Gerhard Weikum: Robust Disambiguation of Named Entities in Text. In \textit{Proceedings of the 2011 Conference on Empirical Methods in Natural Language Processing (EMNLP)}, pages 782--792, Edinburgh, UK, 2011.	

\bibitem[Hoffart 2011]{Hoffart11b} J. Hoffart, F. Suchanek, K. Berberich, E. Lewis-Kelham, G. de Melo, G. Weikum: YAGO2: Exploring and Querying World Knowledge in Time, Space, Context, and Many Languages. In \textit{Proceedings  of the 20th International Conference on World Wide Web (WWW)}, pages 229--232, Hyderabad, India. 2011.

%%\bibitem [Hoffart 2012]{Hoffart2012} J. Hoffart, F. Suchanek, K. Berberich, G. Weikum: YAGO2: A Spatially and Temporally Enhanced Knowledge Base from Wikipedia. Artificial Intelligence 2012. 
%
%%\bibitem{IBM12} IBM Journal of Research and Development 56(3/4), Special Issue on ''This is Watson'', 2012
%
%%\bibitem[Landis 1977]{Landis77}  J. R. Landis,  G. G. Koch:  The measurement of observer agreement for categorical data in Biometrics. Vol. 33, pp. 159–174, 1977.
%
%\bibitem [Kaplan 2002]{Kaplan2002}R. Kaplan: Politics and the American Press: The Rise of Objectivity, pages 1865-1920, New York, Cambridge University Press, 2002. 
%
%\bibitem[Li 2011]{Li2011} X. Li and W. Meng, C. T. Yu: T-verifier: Verifying truthfulness of fact statements. 
%	In \textit{Proceedings of the International Conference on Data Engineering (ICDE)}, pp. 63-74, 2011.
%
%\bibitem[Lin 2001]{Lin01} Dekang Lin, Patrick Pantel: DIRT: discovery of inference rules from text. KDD 2001
%\bibitem[Liu 2005]{Liu05}B. Liu, M. Hu, J. Cheng: Opinion Observer: analyzing and comparing opinions
%on the Web. In\textit{Proceedings  of the 14th International Conference on World Wide Web (WWW)}, pages 342–351, 2005.
%
%\bibitem[Lotan 2013]{Lotan2013}A. Lotan, A. Stern, I. Dagan
%TruthTeller: Annotating Predicate Truth.  In \textit{Proceedings of Human Language Technologies: Conference of the North American Chapter of the Association of Computational Linguistics (HLT-NAACL)}, pp. 752-757, 2013.
%
%%\bibitem [Mindich 1998] {Mindich1998} D. Mindich: Just the Facts: How ``Objectivity" Came to Define American Journalism. New York. New York University Press. 1998.

\bibitem[Kipper 2008]{Kipper08} Karin Kipper, Anna Korhonen, Neville Ryant, Martha Palmer,
A Large-scale Classification of English Verbs,
Language Resources and Evaluation Journal, 42(1): 21-40, 2008,
data available at \url{http://verbs.colorado.edu/~mpalmer/projects/verbnet/downloads.html}


\bibitem[Michel 2011]{Michel11} Jean-Baptiste Michel, Yuan Kui Shen, Aviva Presser Aiden, Adrian Veres, Matthew K. Gray, The Google Books Team, Joseph P. Pickett, Dale Holberg, Dan Clancy, Peter Norvig, Jon Orwant, Steven Pinker, Martin A. Nowak, Erez Lieberman Aiden: Quantitative Analysis of Culture Using Millions of Digitized Books. \textit{Science}, 331(6014):176–182.

\bibitem[Nakashole 2011]{Nakashole11} N. Nakashole, M. Theobald, G. Weikum: Scalable Knowledge Harvesting with High Precision and High Recall.
		In \textit{Proceedings of the 4th International Conference on
		 Web Search and Web Data Mining (WSDM)}, pages 227--326, Hong Kong, China, 2011.
		 
\bibitem[Nakashole 2013]{Nakashole13} N. Nakashole, T. Tylenda, G. Weikum
		 Fine-grained Semantic Typing of Emerging Entities. In \textit{Proceedings of the 51st Annual Meeting of the Association for Computational Linguistics (ACL)}, pp. 1488-1497, 2013.
		 
\bibitem[Nakashole 2012]{Nakashole12} N. Nakashole, G. Weikum, F. Suchanek: PATTY:  A Taxonomy of Relational Patterns with Semantic Types.
		In \textit{Proceedings of the 2012 Joint Conference on Empirical Methods
               in Natural Language Processing and Computational Natural
               Language Learning (EMNLP-CoNLL)}, pages 1135 -1145, Jeju, South Korea, 2012.
          
%\bibitem [Nastase 2010]{Nastase10} V. Nastase, M. Strube, B. Boerschinger, C\"acilia Zirn, Anas Elghafari: WikiNet: A Very Large Scale Multi-Lingual Concept Network. In \textit{Proceedings of the 7th International Conference on Language Resources and Evaluation(LREC)}, Malta, 2010.
%
%\bibitem[Pang 2004]{Pang04}B. Pang, L. Lee: A Sentimental Education: Sentiment Analysis Using Subjectivity Summarization Based on Minimum Cuts.  In \textit{Proceedings of the 42nd Annual Meeting of the Association for Computational Linguistics (ACL)}, 271-278, 2004.
%
%\bibitem[Pasternack 2010]{Pasternack2010} J. Pasternack, D. Roth: Knowing What to Believe. In \textit{Proceedings the  International Conference on Computational Linguistics (COLING)}, pp. 877-885, Beijing, China. 2010.
%
%\bibitem[Pasternack 2013]{Pasternack2013} J. Pasternack, D. Roth: Latent credibility analysis. In \textit{Proceedings  of the 22nd International Conference on World Wide Web (WWW)}, pp. 1009-1020, 2013.
%
%\bibitem[Riloff 2003]{Riloff03}E. Riloff, J.  Wiebe: Learning Learning extraction patterns for subjective expressions. In \textit{Proceedings of the 2011 Conference on Empirical Methods in Natural Language Processing (EMNLP)}, pages 105–112, 2013.
%
%\bibitem[Recasens 2013] {Recasens2013}M. Recasens, C. Danescu-Niculescu-Mizil, Dan Jurafsky: 
% Linguistic Models for Analyzing and Detecting Biased Language.  In \textit{Proceedings of the 51st Annual Meeting of the Association for Computational Linguistics (ACL)}, pp. 1650-1659, 2013.
 
\bibitem[Niu 2012]{Niu12} Feng Niu, Ce Zhang, Christopher Re, Jude W. Shavlik: DeepDive: Web-scale Knowledge-base Construction using Statistical Learning and Inference. In  the VLDS Workshop, pages 25-28, 2012.


\bibitem[Ritter 2012]{Ritter12} A. Ritter, Mausam, O. Etzioni, S. Clark: Open Domain Event Extraction from Twitter.  In \textit{Proceedings of the 18th ACM SIGKDD International Conference on Knowledge
               Discovery and Data Mining (KDD)}, pages 1104-1112, Beijing, China, 2012.
               
\bibitem[Shahaf 2010]{Shahaf10}  D. Shahaf, Carlos Guestrin: Connecting the dots between news articles.  In \textit{Proceedings of the 16th ACM SIGKDD International Conference on Knowledge
Discovery and Data Mining (KDD)}, pages 623-632, 2010.

\bibitem[Sakaki 2010]{Sakaki10}  T. Sakaki, M. Okazaki, Y.  Matsuo: Earthquake Shakes Twitter Users: Real-time Event Detection by Social Sensors.  In \textit{Proceedings  of the 19th International Conference
 on World Wide Web (WWW)}, pages 851-860, Raleigh, North Carolina, USA, 2010.
 
\bibitem[Suchanek 2007]{Suchanek07} F. M. Suchanek, G. Kasneci, G. Weikum: Yago: a Core of Semantic Knowledge. In \textit{Proceedings  of the 16th International Conference
 on World Wide Web (WWW)} pages, 697-706,  Banff, Alberta, Canada, 2007.

\bibitem[Suchanek 2009]{Suchanek09} F. M. Suchanek, M. Sozio, G. Weikum: SOFIE: A Self-organizing Framework for Information Extraction. In\textit{Proceedings  of the 18th International Conference
on World Wide Web (WWW)}, pages 631--640, Madrid, Spain, 2009.
  
%\bibitem[Talukdar 2012]{Talukdar12} P. P. Talukdar, D. T. Wijaya, Tom M. Mitchell: Acquiring temporal constraints between relations.  In \textit{Proceeding of the 21st ACM International Conference on Information and Knowledge Management}, pages 992-1001, CIKM 2012.
%  
% \bibitem[Turney 2002]{Turney02} P. D. Turney: Thumbs up or thumbs down? Semantic orientation applied to unsupervised classification of reviews. In \textit{Proceedings of the 40th Annual Meeting of the Association for Computational Linguistics (ACL)}, pages 417–424. 2002.
%
%%\bibitem[Venetis 2011]{Venetis11} P. Venetis, A. Halevy, J. Madhavan, M. Pasca, W. Shen, F. Wu, G. Miao, C. Wu:
%%Recovering Semantics of Tables on the Web.  In \textit{Proceedings of the VLDB Endowment},  PVLDB 4(9), pages, 528--538. 2011.
%
\bibitem[Wu 2012]{Wu12} W. Wu, H. Li, H. Wang, K. Zhu:
Probase: A Probabilistic Taxonomy for Text Understanding. In \textit{Proceedings of the International Conference on Management of Data (SIGMOD)}, pages 481--492, Scottsdale, AZ, USA, 2012.
\bibitem[W3C 2013]{W3CRDF04} Word Wide Web Consortium (W3C): RDF Primer. \url{http://www.w3.org/TR/rdf-primer/}, Accessed  2013.

%\bibitem[Wiebe 2004]{Wiebe2004}J.  Wiebe, T. Wilson, R. Bruce, M. Bell, M. Martin. Learning
%subjective language. Computational Linguistics, 30(3):277–308. 2004.
%
%%\bibitem[Wikipedia 2013]{Wikipedia2013}  Wikipedia: Neutral point of view. 
%% \url{:http://en.wikipedia.org/wiki/Wikipedia:Neutral_point_of_view}, Accessed  2013.
%
% 
% \bibitem[Yin 2007]{Yin07}X. Yin, J. Han, P. S. Yu : Truth Discovery with Multiple Conflicting Information Providers on the Web. In \textit{Proceedings of the International Conference on  Knowledge Discovery in Databases (KDD)} , pages1048-1052. 2007.
% 
% \bibitem[Yin 2011]{Yin11} X. Yin, W. Tan: Semi-supervised truth discover. In \textit{Proceedings of the 19th International Conference on World Wide Web (WWW)}, pp. 217-226, 2011.
% 
% \bibitem[Yu 2003]{Yu2003}H. Yu, V. Hatzivassiloglou: Towards Answering Opinion Questions: Separating Facts from Opinions and Identifying the Polarity of Opinion Sentences. In \textit{Proceedings of  the Conference on Empirical Methods in Natural Language Processing}, pages. 129-136, 2003.
% 
%  \bibitem[Zhao 2012]{Zhao2012} B. Zhao, B. I. P. Rubinstein, J. Gemmell,  J. Han: A Bayesian approach to discovering truth from conflicting sources for data integration. In \textit{Proceedings of the VLDB Endowment (PVLDB)}, 5(6):550-561, 2012.
%
  
\end{thebibliography}


\end{document}


